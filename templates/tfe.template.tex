% -------- %
% Preamble %
% -------- %
\documentclass[
    $if(papersize)$
        $papersize$,
    $else$
        a4paper,
    $endif$
    $if(fontsize)$
        $fontsize$,
    $else$
        11pt,
    $endif$
    $for(classoption)$$classoption$,$endfor$
    onecolumn,
    % Draft mode will show Overfull Box Warning and figure are not loaded
    % It also disable hyperref in TOC and elsewhere.
    %draft,
    % titlepage: There is a break after \maketitle
    % notitlepage: There is no break after \maketitle
    titlepage,
    % openany: chapter are not on a new page
    % openright: chapter are always on a new page on the right
    %openany,
    % Set language to have proper translation using varioref
    french,
]{$if(documentclass)$$documentclass$$else$report$endif$}


% -------- %
% Packages %
% -------- %

% Encoding
\usepackage[utf8]{inputenc}
\usepackage[T1]{fontenc}

% Language
\usepackage[frenchb]{babel}

% Hyphenation (meaning word breaking rules)
\usepackage{hyphenat}
$for(hyphenation)$
\hyphenation{$hyphenation$}
$endfor$

% Table of Contents
% Add entry for the bibliography and others in the ToC
\usepackage[nottoc,notlot,notlof]{tocbibind}

% Fonts
%\usepackage{fontspec}
% To get the correct name of the font use :
% `fc-list | grep "Linux Libertine" | grep ".otf"`
%\setmainfont[Ligatures=TeX]{Linux Libertine O}
%\setsansfont[Ligatures=TeX]{Linux Biolinum O}
%\setmonofont{Hack}
\usepackage{libertine}
\usepackage{libertinust1math}
\usepackage{sectsty}
% Use sans-serif font for sections title
\allsectionsfont{\sffamily\bfseries}

% Margins
$if(geometry)$
\usepackage[
    %showframe, % Uncomment to debug page margin
    $for(geometry)$
        $geometry$$sep$,
    $endfor$
]{geometry}
$endif$
\usepackage{marginnote}

% Colors
\usepackage{xcolor}

% Images
\usepackage{graphicx}
\graphicspath{{images/}}
% Prevent images to appear in the wrong section
\usepackage[section]{placeins}

% Paragraph interspace
\usepackage{parskip}
\setlength{\parindent}{1em}
\setlength{\parskip}{6pt plus 2pt minus 1pt}
\usepackage[all]{nowidow}

% Header and footer
\usepackage{fancyhdr}
\definecolor{grey-header}{HTML}{777777}
\fancypagestyle{plain}{
    \fancyhf{}
    \renewcommand{\headrule}{}
    \renewcommand{\headrulewidth}{0pt}
    \fancyfoot[C]{\footnotesize\color{grey-header}\thepage}
}
\fancypagestyle{greytwoside}{
    \renewcommand{\headrule}{\color{grey-header}\hrule}
    \renewcommand{\headrulewidth}{0.4pt}
    \fancyhead[LE]{
        \footnotesize\color{grey-header}\thepage
    }
    \fancyhead[RO]{
        \footnotesize\color{grey-header}\thepage
    }
    \fancyhead[RE]{\footnotesize\slshape\color{grey-header} \leftmark}
    \fancyhead[LO]{\footnotesize\slshape\color{grey-header} \rightmark}
    \fancyfoot[C]{}
}
\fancypagestyle{greytwosidenotes}{
    \renewcommand{\headrule}{\color{grey-header}\hrule}
    \renewcommand{\headrulewidth}{0.4pt}
    \fancyhead[LE]{
        \footnotesize\color{grey-header}\thepage
        \marginnote{
            \footnotesize\centering\color{grey-header}%
            \MakeUppercase{Notes du lecteur}
        }
    }
    \fancyhead[RO]{
        \footnotesize\color{grey-header}\thepage
        \marginnote{
            \footnotesize\centering\color{grey-header}%
            \MakeUppercase{Notes du lecteur}
        }
    }
    \fancyhead[RE]{\footnotesize\slshape\color{grey-header} \leftmark}
    \fancyhead[LO]{\footnotesize\slshape\color{grey-header} \rightmark}
    \fancyfoot[C]{}
}
\fancypagestyle{greyoneside}{
    \renewcommand{\headrule}{\color{grey-header}\hrule}
    \renewcommand{\headrulewidth}{0.4pt}
    \fancyhead[LO]{\footnotesize\slshape\color{grey-header} \leftmark}
    \fancyhead[RO]{\footnotesize\color{grey-header}\thepage}
    \fancyfoot[C]{}
}
\fancypagestyle{greyonesidenotes}{
    \renewcommand{\headrule}{\color{grey-header}\hrule}
    \renewcommand{\headrulewidth}{0.4pt}
    \fancyhead[LO]{\footnotesize\slshape\color{grey-header} \leftmark}
    \fancyhead[RO]{
        \footnotesize\color{grey-header}\thepage
        \marginnote{
            \footnotesize\centering\color{grey-header}%
            \MakeUppercase{Notes du lecteur}
        }
    }
    \fancyfoot[C]{}
}
$if(twoside)$
    \pagestyle{greytwoside}
$else$
    \pagestyle{greyoneside}
$endif$
\renewcommand{\chaptermark}[1]{%
    \markboth{\MakeUppercase{\thechapter.\ #1}}{}
}
\renewcommand{\sectionmark}[1]{%
    \markright{\MakeUppercase{\thesection.\ #1}}
}
\usepackage{emptypage}  % Remove page number on blank pages

% Titles
% Source: https://texblog.org/2012/07/03/fancy-latex-chapter-styles/
\usepackage{titlesec}
\definecolor{grey-title}{HTML}{777777}
\titleformat{\chapter}[hang]{\Huge\sffamily\bfseries}{
    \thechapter\hspace{20pt}\textcolor{grey-title}{|}\hspace{20pt}
}{0pt}{\Huge\sffamily\bfseries}
% Setting level of numbering
\setcounter{secnumdepth}{3}
% Change numbering of sections
\renewcommand{\thesubsection}{\roman{subsection}}
\renewcommand{\thesubsubsection}{\alph{subsubsection})}

% Appendix
\usepackage[title, titletoc]{appendix}
\usepackage{pdfpages}

% Math
\usepackage{amsthm}
\usepackage{amsmath}
\usepackage{amssymb}

% Code
\usepackage{listings} % To show code
\usepackage{listingsutf8} % To show code in utf8
% Listing colors
% From: https://github.com/Wandmalfarbe/pandoc-latex-template
\definecolor{listing-background}{HTML}{F7F7F7}
\definecolor{listing-rule}{HTML}{B3B2B3}
\definecolor{listing-numbers}{HTML}{B3B2B3}
\definecolor{listing-text-color}{HTML}{000000}
\definecolor{listing-keyword}{HTML}{435489}
\definecolor{listing-identifier}{HTML}{435489}
\definecolor{listing-string}{HTML}{00999A}
\definecolor{listing-comment}{HTML}{8E8E8E}
\definecolor{listing-javadoc-comment}{HTML}{006CA9}
\lstset{
    language         = python,
    numbers          = left,
    backgroundcolor  = \color{listing-background},
    basicstyle       = \color{listing-text-color}\small\ttfamily{}\linespread{1.15}, % print whole listing small
    xleftmargin      = 2.7em,
    breaklines       = true,
    frame            = single,
    framesep         = 0.6mm,
    rulecolor        = \color{listing-rule},
    frameround       = ffff,
    framexleftmargin = 2.5em,
    tabsize          = 4,
    numberstyle      = \color{listing-numbers},
    aboveskip        = 1.0em,
    keywordstyle     = \color{listing-keyword}\bfseries,
    classoffset      = 0,
    sensitive        = true,
    identifierstyle  = \color{listing-identifier},
    commentstyle     = \color{listing-comment},
    morecomment      = [s][\color{listing-javadoc-comment}]{/**}{*/},
    stringstyle      = \color{listing-string},
    showstringspaces = false,
}

% Link and references
\usepackage{varioref}
\usepackage{url}
$if(colorizelink)$
\definecolor{link-color}{HTML}{BC0000}
$else$
\definecolor{link-color}{HTML}{000000}
$endif$
\renewcommand\UrlFont{\color{link-color}\rmfamily\itshape}  % Color links
\usepackage[
    bookmarks=true,  % Make the links clickable
    pdfauthor={$author$},
    pdftitle={$title$},
    pdfsubject={$subtitle$},
    pdfkeywords={
        $for(tags)$
            $tags$$sep$,
        $endfor$
    },
]{hyperref}
\hypersetup{
    colorlinks,
    linkcolor=link-color,
    citecolor=link-color,
    urlcolor=link-color,
}
\usepackage{hypcap} % The reference to an img link to the top of the img
\usepackage{bookmark} % Allow possibility to have pdf bookmarks

% ------------ %
% New commands %
% ------------ %

% Blockquote
% From: https://github.com/Wandmalfarbe/pandoc-latex-template
\definecolor{blockquote-border}{RGB}{221,221,221}
\definecolor{blockquote-text}{RGB}{119,119,119}
\usepackage{mdframed}
\newmdenv[
    rightline=false,
    bottomline=false,
    topline=false,
    linewidth=3pt,
    linecolor=blockquote-border,
    skipabove=\parskip,
]{customblockquote}
\renewenvironment{quote}
{
    \begin{customblockquote}
    \list{}{\rightmargin=0em\leftmargin=0em}%
    \item\relax\color{blockquote-text}\ignorespaces
}
{
    \unskip\unskip\endlist
    \end{customblockquote}
}

% Caption
% From: https://github.com/Wandmalfarbe/pandoc-latex-template
\definecolor{caption-color}{HTML}{777777}
\usepackage{setspace}  % Allow to set interline spaces
\usepackage[
    font={stretch=1.2},
    textfont={color=caption-color},
    position=top,
    skip=4mm,
    labelfont=bf,
    singlelinecheck=false,
    justification=justified]{caption}
\captionsetup[longtable]{position=above}


% Command defined by pandoc
\setlength{\emergencystretch}{3em}  % prevent overfull lines
\providecommand{\tightlist}{%
    \setlength{\itemsep}{0pt}\setlength{\parskip}{0pt}}


% ------------ %
% The document %
% ------------ %

\begin{document}

    % --------- %
    % Half page %
    % --------- %

    $if(paperversion)$
    \begin{titlepage}
        \label{halfpage}
        \pdfbookmark{Half Page}{halfpage}  % To have pdf bookmarks
        \mbox{}
        \sffamily
        \vspace{\stretch{1}}
        \begin{center}
            {\Large
                $title$ \par
            }
            \vspace{\baselineskip}
            {\normalsize
                $subtitle$
            }
        \end{center}
        \vspace{\stretch{1}}
    \end{titlepage}
    \cleardoublepage
    $endif$


    % -------------------- %
    % Title page variables %
    % -------------------- %

    $if(titlepage)$
        \label{titlepage}
        \pdfbookmark{Title Page}{titlepage}  % To have pdf bookmarks
        \includepdf{$titlepage$}
        $if(licence)$
        $else$
            \cleardoublepage
        $endif$
    $endif$


    % ------- %
    % Licence %
    % ------- %

    $if(licence)$
        \label{licence}
        \pdfbookmark{Licence}{licence}  % To have pdf bookmarks
        \thispagestyle{empty}
        \input{$licence$}
        \newpage
    $endif$

    \pagenumbering{roman}  % Reset page number to roman style

    $if(abstract)$
    % Uncomment the 2 next lines and comment the last one if you don't
    % want the abstract to appear in the table of contents.
    %\label{abstract}
    %\pdfbookmark{\abstractname}{abstract}  % To have pdf bookmarks
    \phantomsection\addcontentsline{toc}{chapter}{\abstractname}
    \thispagestyle{plain}
    \section*{\abstractname}
    \input{$abstract$}
    \cleardoublepage
    $endif$


    % -------------- %
    % Specifications %
    % -------------- %

    $if(specifications)$
        \phantomsection\addcontentsline{toc}{chapter}{Cahier des charges}
        \includepdf[
            scale=1,
            pages=-,
            landscape=false,
        ]{$specifications$}
        \cleardoublepage
    $endif$


    % ----------------- %
    % Table of Contents %
    % ----------------- %

    \sffamily
    \label{tableofcontents}
    \pdfbookmark{\contentsname}{tableofcontents}  % To have pdf bookmarks
    \setcounter{tocdepth}{2}
    \tableofcontents
    \normalfont

    \cleardoublepage

    \pagenumbering{arabic}  % Reset page number to arabic style


    % -------------------- %
    % Core of the document %
    % -------------------- %

    $if(side_notes)$
        \newgeometry{
            $for(side_notes)$
                $side_notes$,
            $endfor$
        }
        $if(twoside)$
        \pagestyle{greytwosidenotes}
        $else$
        \pagestyle{greyonesidenotes}
        $endif$
        \fancypagestyle{plain}{
            \renewcommand{\headrule}{}
            \renewcommand{\headrulewidth}{0pt}
            \fancyhead{
                \marginnote{
                    \footnotesize\normalfont\centering\color{grey-header}%
                    \MakeUppercase{Notes du lecteur}
                }[115pt]
            }
            \fancyfoot[C]{\footnotesize\color{grey-header}\thepage}
        }
    $endif$

    $body$

    $if(side_notes)$
    \restoregeometry
        $if(twoside)$
            \pagestyle{greytwoside}
        $else$
            \pagestyle{greyoneside}
        $endif$
        \fancypagestyle{plain}{
            \fancyhead{}
            \renewcommand{\headrule}{}
            \renewcommand{\headrulewidth}{0pt}
            \fancyfoot[C]{\footnotesize\color{grey-header}\thepage}
        }
    $endif$


    % ------------ %
    % Bibliography %
    % ------------ %

    $if(bibfile)$
        \bibliographystyle{plain}
        \bibliography{$bibfile$}
    $endif$


    % ---------- %
    % Appendices %
    % ---------- %
    $if(appendices)$
        \clearpage
        \begin{appendices}
            \input{$appendices$}
        \end{appendices}
    $endif$


    % -------- %
    % End page %
    % -------- %
    $if(endpage)$
        \newpage
        \thispagestyle{empty}
        \input{$endpage$}
    $endif$

\end{document}
